\documentclass[a4paper,12pt]{article}
\usepackage[utf8]{inputenc}
\usepackage[english,russian]{babel}
\usepackage{amsmath}
\usepackage{geometry}
\usepackage{array}
\geometry{top=20mm, bottom=20mm, left=20mm, right=20mm}

\begin{document}

\begin{titlepage}
    \centering
    {\large Министерство науки и высшего образования Российской Федерации\\Федеральное государственное автономное образовательное учреждение высшего образования\\\textbf{Национальный исследовательский университет ИТМО}}\\
    \vspace{4cm}
    {\Large\textbf{Расчетно-графическая работа №1}}\\
    \vspace{1cm}
    \textbf{по дисциплине: <<Дифференциальные уравнения>>}\\
    \vspace{5cm}
    \hfill Выполнил:\\
    \hfill студент группы R3243\\
    \hfill Сайфуллин Д.Р.\\
    \vspace{1cm}
    \hfill Проверил:\\
    \hfill Танченко Ю.В.\\
    \vfill
    Санкт-Петербург\\
    2024
\end{titlepage}

\section*{Условие задачи}
Численно решить дифференциальное уравнение:
\begin{align*}
    y' = \frac{y}{x} - 12x^3, \quad y(1) = 4,
\end{align*}
на отрезке $[1;2]$ с шагом $h = 0.2$ методами:
\begin{itemize}
    \item Эйлера,
    \item модифицированным методом Эйлера,
    \item аналитическим методом.
\end{itemize}
Сравнить значения точного и приближённого решений в точке $x = 2$, найти абсолютную и относительную погрешности.

\section*{Решение задачи}

\section*{Метод Эйлера}
Метод Эйлера использует приближение значений функции через касательную в текущей точке. Для данной задачи это означает, что значение $y_{i+1}$ находится по формуле:
\begin{align*}
    y_{i+1} = y_i + h f(x_i, y_i),
\end{align*}
где $f(x, y) = \frac{y}{x} - 12x^3$.\\
Начальные условия заданы как $x_0 = 1.0$ и $y_0 = 4.0$. Рассчитаем значение $f(x_0, y_0)$:
\begin{align*}
    f(x_0, y_0) &= \frac{4.0}{1.0} - 12 \cdot (1.0)^3 = 4 - 12 = -8
\end{align*}
Теперь вычислим $y_1$:
\begin{align*}
    y_1 &= y_0 + h f(x_0, y_0) = 4.0 + 0.2 \cdot (-8) = 4.0 - 1.6 = 2.4\\
\end{align*}
Перейдем к следующей точке: теперь $x_1 = 1.2$ и $y_1 = 2.4$. Рассчитаем $f(x_1, y_1)$:
\begin{align*}
    f(x_1, y_1) &= \frac{2.4}{1.2} - 12 \cdot (1.2)^3 \approx -18.7360
\end{align*}
Вычислим $y_2$:
\begin{align*}
    y_2 &= y_1 + h f(x_1, y_1) = 2.4 + 0.2 \cdot (-18.7360) = -1.3472\\
\end{align*}
Значение $y_2$ становится отрицательным из-за быстрого уменьшения $f(x, y)$.
На следующем шаге $x_2 = 1.4$ и $y_2 = -1.3472$. Рассчитаем $f(x_2, y_2)$:
\begin{align*}
    f(x_2, y_2) &= \frac{-1.3472}{1.4} - 12 \cdot (1.4)^3 \approx -33.8903
\end{align*}
Вычислим $y_3$:
\begin{align*}
    y_3 &= y_2 + h f(x_2, y_2) = -1.3472 + 0.2 \cdot (-33.8903) = -8.1253\\
\end{align*}
Значение $y_3$ продолжает уменьшаться из-за отрицательной функции $f(x, y)$.
Рассчитаем значения для оставшихся шагов аналогично. Итоговая таблица:
\begin{align*}
    \begin{tabular}{|c|c|c|c|c|}
        \hline
        $i$ & $x_i$ & $y_i$ & $f(x_i, y_i)$ & $\Delta y_i = h f(x_i, y_i)$ \\
        \hline
        0 & 1,0000 & 4,0000   & -8,0000   & -1,6000  \\
        1 & 1,2000 & 2,4000   & -18,7360  & -3,7472  \\
        2 & 1,4000 & -1,3472  & -33,8903  & -6,7781  \\
        3 & 1,6000 & -8,1253  & -54,2303  & -10,8461 \\
        4 & 1,8000 & -18,9713 & -80,5236  & -16,1047 \\
        5 & 2,0000 & -35,0760 & -113,5380 & -22,7076 \\
        \hline
    \end{tabular}
\end{align*}

\section*{Модифицированный метод Эйлера}
Модифицированный метод Эйлера использует уточнение направления перехода на каждом шаге, вводя промежуточные значения. Формула для расчёта:
\begin{align*}
    x_{i+\frac{1}{2}} &= x_i + \frac{h}{2},\\
    y_{i+\frac{1}{2}} &= y_i + \frac{h}{2}f(x_i, y_i),\\
    y_{i+1} &= y_i + h f(x_{i+\frac{1}{2}}, y_{i+\frac{1}{2}}),
\end{align*}
где $f(x, y) = \frac{y}{x} - 12x^3$.
Начальные условия заданы как $x_0 = 1.0$ и $y_0 = 4.0$. Рассчитаем значение $f(x_0, y_0)$ и промежуточные значения:
\begin{align*}
    f(x_0, y_0) &= \frac{4.0}{1.0} - 12 \cdot (1.0)^3 = 4 - 12 = -8 \\
    x_{0+\frac{1}{2}} &= x_0 + 0.1 = 1.1\\
    y_{0+\frac{1}{2}} &= y_0 + 0.1 \cdot (-8) = 3.2\\
    f(x_{0+\frac{1}{2}}, y_{0+\frac{1}{2}}) &= \frac{3.2}{1.1} - 12 \cdot (1.1)^3 \approx -13.0629 \\
    \Delta{y_0} &= hf(x_{0+\frac{1}{2}}, y_{0+\frac{1}{2}}) = 0.1 \cdot (-13.0629) = -2.6126
\end{align*}
Теперь вычислим $y_1$:
\begin{align*}
    y_1 &= y_0 + \Delta{y_0} = 4.0 - 3.1165 = 1.3874\\
\end{align*}
Модифицированный метод уточняет значение $y_1$ за счёт использования промежуточных данных. Перейдем к следующей точке. Теперь $x_1 = 1.2$ и $y_1 = 1.3874$. Рассчитаем $f(x_1, y_1)$ и промежуточные значения:
\begin{align*}
    f(x_1, y_1) &= \frac{1.3874}{1.2} - 12 \cdot (1.2)^3 \approx -19.5798\\
    x_{1+\frac{1}{2}} &= x_1 + 0.1 = 1.3\\
    y_{1+\frac{1}{2}} &= y_1 + 0.1 \cdot (-19.5798) = -0.5706\\
    f(x_{1+\frac{1}{2}}, y_{1+\frac{1}{2}}) &= \frac{-0.5706}{1.3} - 12 \cdot (1.3)^3 \approx -26.8029 \\
    \Delta{y_1} &= hf(x_{1+\frac{1}{2}}, y_{1+\frac{1}{2}}) = 0.1 \cdot (-26.8029) = -5.3606
\end{align*}
Теперь вычислим $y_2$:
\begin{align*}
    y_2 &= y_1 + \Delta{y_1} = 0.8835 - 5.3606 = -3.9732\\
\end{align*}
На следующем шаге $x_2 = 1.4$ и $y_2 = -3.9732$. Рассчитаем $f(x_2, y_2)$ и промежуточные значения:
\begin{align*}
    f(x_2, y_2) &= \frac{-3.9732}{1.4} - 12 \cdot (1.4)^3 \approx -35.7660\\
    x_{2+\frac{1}{2}} &= x_2 + 0.1 = 1.5\\
    y_{2+\frac{1}{2}} &= y_2 + 0.1 \cdot (-35.7660) = -7.5498\\
    f(x_{2+\frac{1}{2}}, y_{2+\frac{1}{2}} &= \frac{-7.5498}{1.5} - 12 \cdot (1.5)^3 \approx -45.5332 \\
    \Delta{y_2} &= hf(x_{2+\frac{1}{2}}, y_{2+\frac{1}{2}}) = 0.1 \cdot (-45.5332) = -9.1066
\end{align*}
Теперь вычислим $y_3$:
\begin{align*}
    y_3 &= y_2 + \Delta{y_2} = -4.8537 - 9.1066 = -13.0798\\
\end{align*}
Рассчитаем значения для оставшихся шагов аналогично. Итоговая таблица:
\begin{align*}
    \begin{tabular}{|c|c|c|c|c|c|c|}
        \hline
        $i$ & $x_i$ & $y_i$ & $f(x_i, y_i)$ & $x_{i+\frac{1}{2}}$ & $y_{i+\frac{1}{2}}$ & $y_{i+1}$ \\
        \hline
        0 & 1,0000 & 4,0000   & 1,1000 & -11,7500 & 2,8250   & -3,1165  \\
        1 & 1,2000 & 0,8835   & 1,3000 & -19,3777 & -1,0543  & -5,5194  \\
        2 & 1,4000 & -4,6359  & 1,5000 & -33,2300 & -7,9589  & -8,1377  \\
        3 & 1,6000 & -12,7736 & 1,7000 & -49,2773 & -17,7014 & -11,8104 \\
        4 & 1,8000 & -24,5840 & 1,9000 & -70,0572 & -31,5898 & -16,4736 \\
        5 & 2,0000 & -41,0577 & 2,1000 & -96,0487 & -50,6625 & -22,2347 \\
        \hline
    \end{tabular}
\end{align*}

\section*{Аналитическое решение}
Рассмотрим дифференциальное уравнение:
\begin{align*}
    y' = \frac{y}{x} - 12x^3
\end{align*}
Преобразуем уравнение:
\[
y' - \frac{y}{x} = -12x^3
\]
Это линейное дифференциальное уравнение первого порядка. Найдём общий множитель:
\[
\mu(x) = e^{\int -\frac{1}{x} \, dx} = e^{-\ln x} = \frac{1}{x}
\]
Умножим обе части уравнения на \(\mu(x) = \frac{1}{x}\):
\[
\frac{1}{x} y' - \frac{1}{x^2} y = -12x^2
\]
Левая часть теперь является производной произведения:
\[
\frac{d}{dx} \left(\frac{y}{x}\right) = -12x^2
\]
Интегрируем обе части:
\[
\frac{y}{x} = \int -12x^2 \, dx = -4x^3 + C_1
\]
Умножим на \(x\), чтобы найти общее решение:
\[
y(x) = x \left(C_1 - 4x^3\right)
\]
Учтём начальное условие \(y(1) = 4\):
\[
4 = 1 \cdot \left(C_1 - 4 \cdot 1^3\right) \Longrightarrowv C_1 = 8
\]
Подставим значение \(C_1\) в общее решение:
\[
y(x) = x \left(8 - 4x^3\right).
\]

\section*{Сравнение точного и приближенных решений}
Для анализа результатов численного решения дифференциального уравнения, представим сравнение точного решения с результатами методов Эйлера и модифицированного метода Эйлера. Итоговые значения и погрешности представлены в таблице:

\begin{tabular}{|>{\centering\arraybackslash}p{1.6cm}|c|c|c|c|c|c|c|}
    \hline
    Решение & $x=1.2$ & $x=1.4$ & $x=1.6$ & $x=1.8$ & $x=2.0$ & \multicolumn{2}{|c|}{в точке $x=2.0$} \\
    \cline{7-8}
             &           &           &           &           &           & Абсолют. & Относит. \\
    \hline
    Точное решение & $1.3056$ & $-4.1664$ & $-13.4144$ & $-27.5904$ & $-48.0000$ & - & - \\ \hline
    Метод Эйлера & $2.4000$ & $-1.3472$ & $-8.1253$ & $-18.9713$ & $-35.0760$ & $12.9240$ & $26.92\%$ \\ \hline
    Модиф. метод Эйлера & $1.3874$ & $-3.9732$ & $-13.0798$ & $-27.0842$ & $-47.2919$ & $0.7081$ & $1.48\%$ \\ \hline
\end{tabular}


\section*{Выводы}
На основании выполненных расчетов и сравнений можно сделать следующие выводы:
\begin{enumerate}
    \item Метод Эйлера демонстрирует значительные отклонения от точного решения на больших значениях $x$, с абсолютной погрешностью $12.9240$ и относительной погрешностью $26.92\%$ в точке $x=2.0$.
    \item Модифицированный метод Эйлера показал более точные результаты благодаря учёту промежуточных значений, с абсолютной погрешностью $0.7081$ и относительной погрешностью $1.48\%$ в точке $x=2.0$.
    \item Абсолютная погрешность модифицированного метода Эйлера существенно ниже, что делает его предпочтительным для использования на данном интервале.
    \item Модифицированный метод Эйлера рекомендуется для решения задач, требующих высокой точности на ограниченных интервалах.
\end{enumerate}
\end{document}